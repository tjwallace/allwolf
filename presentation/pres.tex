\documentclass{beamer}
\usetheme[height=10mm]{Rochester} 
% \usetheme{Boadilla} 
% \usetheme{default}
% \usepackage{beamerthemesplit}

\usepackage[english]{babel}
\usepackage{alltt}  % so you can include program text 
\usepackage{listings}
\usepackage[utf8]{inputenc}
\usepackage[small]{caption}
% \usepackage{color}

\lstset{
basicstyle=\tiny\ttfamily, % code font size 
numbers=none, % where to put the line-numbers 
stepnumber=1, % the step between two line-numbers. 
frame=single, 
frameround=tttt,
captionpos=b, % t or b 
showstringspaces=false, % underline spaces within strings 
showspaces=false, % show spaces within strings with underscores 
showtabs=false, % show tabs within strings with underscores 
breaklines=true, % Break long lines of code 
tabsize=2, % sets default tabsize to 4 spaces
escapeinside={(*@}{@*)}, % To add labels in listing, needs own line
keywordstyle=\color{blue}, %colors
commentstyle=\color{green},
stringstyle=\color{red}
} 

\title{Allwolf Implementation}
\author{T. Jeff Wallace}
\institute{McGill University}
\date{\today}

\begin{document}

\frame{\titlepage}

\section[Outline]{}
\begin{frame}
	\frametitle{Outline}
	\tableofcontents
\end{frame}

\section{Concurrency}

\subsection{ConcurrentHashMap}
\begin{frame}
	\frametitle{Shared Memory with {\tt ConcurrentHashMap}}
	\begin{definition}
	A hash table supporting full concurrency of retrievals and adjustable expected concurrency for updates.
	\end{definition}
	
	\begin{description}
		\item[How does it work?] \hfill \\
		The table is internally partitioned to try to permit the indicated number of concurrent updates without contention.
		\item[Why use it?] \hfill \\
		More concurrency.
	\end{description}
\end{frame}

\subsection{CyclicBarrier}
\begin{frame}[fragile]
	\frametitle{{\tt CyclicBarrier}}
	\begin{definition}
	A synchronization aid that allows a set of threads to all wait for each other to reach a common barrier point.
	\end{definition}

	\begin{lstlisting}[language=Java,caption={Initializing the {\tt CyclicBarrier} in {\tt Game.java}}]
	barrier = new CyclicBarrier(NUM_OF_SHEEP + NUM_OF_WOLVES, new EndGameCheck());
	\end{lstlisting}
	
	\begin{lstlisting}[language=Java,caption={Using the {\tt CyclicBarrier} in {\tt Agent.java}}]
	public void run()
	{
	    while(alive)
	    {
	        board.moveAgent(this, nextPos());
	        barrier.await();
	    }
	}
	\end{lstlisting}
	
	\begin{itemize}
		\item Assert that an agent moves only {\em once} per cycle.
		\item Check for {\em end game} condition after each cycle using the inner class {\tt EndGameCheck}
	\end{itemize}
\end{frame}

\section{Extensions}

\subsection{Agent Speed}
\begin{frame}
	\frametitle{Variable Agent Speed}	
	\begin{description}
		\item[Description] \hfill \\
		Allow each agent type to move an arbitrary distance in {\em north}, {\em south}, {\em east}, or {\em west} direction(s) per cycle.
		\item[Concurrency problems] \hfill \\
		Agent computes next position but another agent has already moved there.
	\end{description}
\end{frame}

\end{document}
